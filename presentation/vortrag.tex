


\documentclass[compress,t]{beamer}

\usetheme{hsrm}


\usepackage[utf8]{inputenc}
\usepackage[german]{babel}
\usepackage{graphicx}
\usepackage{multicol}
\usepackage{tabularx,ragged2e}
\usepackage{booktabs}
\usepackage{listings}

\usepackage{hhline}
\usepackage{multirow}

\usepackage{color}

%--------------------------------------------------------------------------
% General presentation settings
%--------------------------------------------------------------------------
\title{Privacy Ranking}
\subtitle{Wahlprojekt SS 2017}
\date{Letztes Update: \today}
\author{Max Mustermann}
\institute{Studienbereich Informatik\\Hochschule {RheinMain}}

\lstset{
  frame=single,
  basicstyle={\small\ttfamily},
  aboveskip=3mm,
  belowskip=3mm,
  breaklines=true,
  breakatwhitespace=true,
  tabsize=4,
  escapeinside={(*@}{@*)}
}

\begin{document}

\maketitle


%--------------------------------------------------------------------------
% Table of contents
%--------------------------------------------------------------------------
\section*{Gliederung}
\begin{frame}{Gliederung}
    \tableofcontents[hideallsubsections]
\end{frame}


%--------------------------------------------------------------------------
% Content
%--------------------------------------------------------------------------

\section{Einleitung}

\begin{frame}{Einleitung}

    \begin{itemize}
        \item Ziel des Projekts?
        \item Architektur (App - Web - Datenbank)
        \item Live-Demo der App
    \end{itemize}

\end{frame}

\begin{frame}{Ziel des Projekts?}

    \begin{itemize}
        \item Scraper
        \item Database
        \item Clustering
        \item Scoring
        \item Webservice
        \item App Erstellung
    \end{itemize}

\end{frame}

\begin{frame}{Architektur}

    \begin{center}
        \includegraphics[width=0.9\textwidth]{img/architecture.png}
    \end{center}

\end{frame}

\begin{frame}{Live-Demo}

\end{frame}

\section{App Beispielcode}

\begin{frame}
    Unsere App halt
\end{frame}

\section{Webservice}

\begin{frame}{Webservice}

    \begin{center}
        \begin{itemize}
            \item Was ist ein Webservice?
            \item Warum wird er in diesem Projekt benötigt?
            \item Representational State Transfer (REST)
        \end{itemize}
    \end{center}

\end{frame}

\begin {frame}{Slim Framework}

    \begin{center}
        \begin{itemize}
            \item Was ist Slim?
            \item Warum nicht \"from scratch\" selbst coden?
            \item Hat das auch Nachteile?
        \end{itemize}
    \end{center}

\end{frame}

\begin {frame}[fragile]{Datenbankverbindung}

    \begin{center}
        \begin{lstlisting}
$config['db']['host']   = "localhost";
$config['db']['user']   = "XXXXXXXXXXXXX";
$config['db']['pass']   = "XXXXXXXXXXXXX";
$config['db']['dbname'] = "privacy_ranking";

$app = new \Slim\App(["settings" => $config]);
$container = $app->getContainer();
        \end{lstlisting}
    \end{center}

\end{frame}

\begin {frame}[fragile]{Datenbankverbindung}

    \begin{center}
        \begin{lstlisting}
$container['db'] = function ($c) {
    $db = $c['settings']['db'];
    $pdo = new PDO("mysql:host=" . $db['host'] . ";dbname=" . $db['dbname'].";charset=utf8",
        $db['user'], $db['pass']);
    $pdo->setAttribute(PDO::ATTR_ERRMODE, PDO::ERRMODE_EXCEPTION);
    $pdo->setAttribute(
    PDO::ATTR_DEFAULT_FETCH_MODE, PDO::FETCH_ASSOC);
    return $pdo;
};

        \end{lstlisting}
    \end{center}

\end{frame}

\begin {frame}[fragile]{Beispiel Anfrage}

    \begin{center}
        \begin{lstlisting}

$app->get('/perm/[{id}]', function ($request, $response, $args) {

    try
    {

        $sth = $this->db->prepare("SELECT name, Permission_id, weight  FROM Apps NATURAL JOIN App_permissions NATURAL JOIN Permissions  WHERE App_id=:id");

        $sth->bindParam("id",$args['id']);

        $sth->execute();

        \end{lstlisting}
    \end{center}

\end{frame}

\begin {frame}[fragile]{Beispiel Anfrage}

    \begin{center}
        \begin{lstlisting}
        $category = $sth->fetchAll();

        if($category) {
            return $this->response->withJson(
            $category, 200);

        } else {
            throw new PDOException('"No Permissions needed."');
        }

    } catch(PDOException $e) {
          echo '[{"name":'. $e->getMessage() .'}]';
    }
});
        \end{lstlisting}
    \end{center}

\end{frame}

\begin {frame}{Beispiel Anfrage}

    \begin{center}
         \begin{itemize}
            \item Anfrage an http://privacyranking.cs.hs-rm.de/perm/com.tinder wird gestellt.
            \item nginx leitet an Slim weiter
            \item Slim ruft get('/perm/[{id}]'... auf
            \item DB Anfrage wird vorbereitet und ausgeführt
            \item ergebniss wird als JSON gepackt zurückgegeben
        \end{itemize}
    \end{center}

\end{frame}

\begin {frame}[fragile]{Beispiel Anfrage Antwort}

    \begin{center}
            \begin{lstlisting}
[{"name":"In-App-K\u00e4ufe","Permission_id":"0",
"weight":"0.1"},
{"name":"Ger\u00e4te- & App-Verlauf","Permission_id":"1",
"weight":"0.7"},
{"name":"Standord","Permission_id":"6",
"weight":"1"},
{"name":"Telefon","Permission_id":"8",
"weight":"0.7"},
(...)]
            \end{lstlisting}
    \end{center}

\end{frame}

\begin {frame}{Swagger}

    \begin{center}
         "The OpenAPI Specification (OAS)[formerly known as the Swagger Specification] defines a standard, language-agnostic interface to RESTful APIs which allows both humans and computers to discover and understand the capabilities of the service without access to source code, documentation, or through network traffic inspection. When properly defined, a consumer can understand and interact with the remote service with a minimal amount of implementation logic."

    \end{center}

\end{frame}

\begin {frame}{Swagger}

    \begin{center}
         swagger bilder hinzufügen TODO

    \end{center}

\end{frame}
\section{Datenbeschaffung und Verarbeitung}

\begin{frame}{Website Google Playstore}

    \begin{center}
        \includegraphics[width=0.6\textwidth]{img/google_play.png}
        \newline
        \includegraphics[width=0.6\textwidth]{img/google_play_network.png}
    \end{center}

\end{frame}

\begin{frame}[fragile]{Scraping der Daten}

    \begin{itemize}
        \item Zugriff auf den Webservice von Google
        \item https://play.google.com/store/xhr/getdoc?authuser=0
        \item POST (ids=app\_id, xhr=1)
    \end{itemize}

    \begin{lstlisting}
[["gdar",1,[["me.pou.app","me.pou.app",1,3,
"/store/apps/details?id\u003dme.pou.app",
"/store/apps/details?id\u003dme.pou.app",
"https://play.google.com/store/apps/details
?id\u003dme.pou.app","https://market.android
.com/details?id\u003dme.pou.app","Pou",...
    \end{lstlisting}

\end{frame}

\begin{frame}[fragile]{Extrahieren der Daten}
    \begin{itemize}
        \item Schreiben eines Wrappers in Python
        \item Lokalisieren der nötigen Informationen
    \end{itemize}

    \begin{lstlisting}[language=Python]
def extract_title(data):
    return _remove_emojis(data[0][2][0][8])

def extract_description(data):
    return _remove_emojis(data[0][2][0][9])

def extract_rating(data):
    return data[0][2][0][23]
    \end{lstlisting}

\end{frame}

\section{Datenbank}

\begin{frame}{MariaDB Datenbank}

    \begin{center}
        \includegraphics[width=0.9\textwidth]{img/Scheme_extended_3.png}
    \end{center}

\end{frame}

\section{Kategorisierung und Bewertung der Apps}

\begin{frame}{Datamining}

    \begin{itemize}
        \item Kategorisierung mithilfe von Clustering
        \item Auswahl zwischen den einzelnen Algorithmen
            \begin{itemize}
                \item K-Means
                    \begin{itemize}
                        \item Anzahl Cluster muss bekannt sein
                    \end{itemize}
                \item Affinity propagation
                    \begin{itemize}
                        \item Terminiert nicht
                    \end{itemize}
                \item Mean-Shift
                    \begin{itemize}
                        \item Terminiert nicht
                    \end{itemize}
                \item Ward hierarchical clustering
                    \begin{itemize}
                        \item Terminiert nicht
                    \end{itemize}
                \item DBSCAN
                    \begin{itemize}
                        \item Rauschen
                    \end{itemize}
            \end{itemize}
    \end{itemize}

\end{frame}

\begin{frame}{DBSCAN}

    \begin{center}
        \includegraphics[width=0.4\textwidth]{img/dbscan.png}{Quelle: \href{http://www.oracle.com/technetwork/java/javase/tech/memorymanagement-whitepaper-1-150020.pdf}{Wikipedia}}
    \end{center}

    \begin{itemize}
        \item Density-based spatial clustering of applications with noise
        \item Abstand (Epsilon) muss gut gewählt werden
    \end{itemize}

\end{frame}

\begin{frame}{TF-IDF}

    \begin{itemize}
        \item Clustering-Algorithmen funktionieren nur mit numerischen Werten
        \item Text frequenzy
            \begin{itemize}
                \item Je häufiger Wort in Text enthalten \(\Rightarrow\) bedeutend
                \item Wert für \emph{min-df} muss gut gewählt werden
            \end{itemize}
        \item Inversed document frequenzy
            \begin{itemize}
                \item Je häufiger Wort in allen Dokumenten enthalten \(\Rightarrow\) unbedeutend
                \item Wert für \emph{max-df} muss gut gewählt werden
            \end{itemize}
        \item Dadurch entsteht Documents \(\times\) Features Matrix
        \item Max. Feautures werden bestimmt.
    \end{itemize}

\end{frame}

\begin{frame}{Gute Metric finden}

    \begin{columns}[t]
        \column{.5\textwidth}
        \centering
        \includegraphics[width=0.8\textwidth]{img/1_euclidian.png}{Euclidian}\\
        \includegraphics[width=0.8\textwidth]{img/2_cosine_brute.png}{Cosine}
        \column{.5\textwidth}
        \centering
        \includegraphics[width=0.8\textwidth]{img/3_l2.png}{L2}\\
        \includegraphics[width=0.8\textwidth]{img/4_minkowski_p_2_0.png}{Minkowski}
    \end{columns}

\end{frame}

\begin{frame}{Gute Parameter finden - Testdaten}

    \begin{columns}[t]
        \column{.5\textwidth}
        \centering
        \includegraphics[width=1.0\textwidth]{img/df_3_cosin.png}\\
        \includegraphics[width=1.0\textwidth]{img/df_eps_1_cosin.png}
        \column{.5\textwidth}
        \centering
        \includegraphics[width=1.0\textwidth]{img/df_3_cosin_nofc.png}\\
        \includegraphics[width=1.0\textwidth]{img/df_eps_1_cosin_nofc.png}
    \end{columns}

\end{frame}

\begin{frame}{Gute Parameter finden - Google Play Daten}

    \begin{columns}[t]
        \column{.5\textwidth}
        \centering
        \includegraphics[width=1.0\textwidth]{img/gp_2_cosin.png}\\
        \includegraphics[width=1.0\textwidth]{img/features_3.png}
        \column{.5\textwidth}
        \centering
        \includegraphics[width=1.0\textwidth]{img/gp_2_cosin_nofc.png}\\
        \includegraphics[width=1.0\textwidth]{img/features_3_nofc.png}
    \end{columns}

\end{frame}

\begin{frame}{Gute Parameter finden}

    \begin{itemize}
        \item max-df: 0.01
        \item min-df: 0.005
        \item eps: 0.45
        \item min-samples: 30
        \item features: 1500
    \end{itemize}

    \textcolor{green}{\(\Rightarrow\)} 42 Cluster\\
    \textcolor{red}{\(\Rightarrow\)} Mehr als 50\% Rauschen\\
    \textcolor{red}{\(\Rightarrow\)} 1 Cluster viel zu groß

\end{frame}

\begin{frame}{Kombination mit anderen Algorithmen}

    \begin{itemize}
        \item K-Means
            \begin{itemize}
                \item Anzahl Cluster aus DBSCAN \(\rightarrow\) \textcolor{orange}{mäßiger} Erfolg
                \item Anzahl GP Kategorien \(\rightarrow\) \textcolor{orange}{mäßiger} Erfolg
            \end{itemize}
        \item Classifier
            \begin{itemize}
                \item DecisionTree \(\rightarrow\) \textcolor{red}{miserabler} Erfolg
                \item BernoulliNB \(\rightarrow\) \textcolor{red}{miserabler} Erfolg
                \item MLP \(\rightarrow\) \textcolor{red}{miserabler} Erfolg
                \item AdaBoost \(\rightarrow\) \textcolor{red}{miserabler} Erfolg
                \item KNeighbors  \(\rightarrow\) \textcolor{green}{akzeptabler} Erfolg
            \end{itemize}
    \end{itemize}

    \textcolor{green}{\(\Rightarrow\)} Kein Verlust mehr durch Rauschen\\
    \textcolor{red}{\(\Rightarrow\)} Zu großer Cluster wurde noch größer\\
    \textcolor{red}{\(\Rightarrow\)} Cluster beinhaltet mehr als 50\% apps

\end{frame}

\begin{frame}{Hierarchical DBSCAN}

    Aufteilung von zu großen Cluster in kleinere.

    \textcolor{red}{\(\Rightarrow\)} Sprengt den Arbeitsspeicher.
    \begin{quote}
        Dies liegt an der mieserablen Implementierung in SKLearn.
        Es ist besser, wenn du's selbst implementierst.\\
        - Viele Leute bei Stackoverflow
    \end{quote}

    Eigene Variante in Kombination mit KNeighbors:
    \begin{itemize}
        \item Zu große Cluster werden erneut mit DBSCAN geclustert (kleineres Epsilon)
        \item Dabei entstandendes Rauschen wird mithilfe KNeighbors neu verteilt
    \end{itemize}

    \textcolor{red}{\(\Rightarrow\)} Clusterqualität wurde schlechter, kein guter Erfolg

\end{frame}

\begin{frame}{Bewertung der Apps}

    Die Apps werden nach dem Einfluss auf die Privatsphäre bewertet.

    \begin{enumerate}
        \item Sammeln der Berechtigungen innerhalb eines Clusters
    \end{enumerate}

    \begin{center}
    \begin{tabular}{|*{6}{c|}}
        \multicolumn{6}{c}{Permissions} \tabularnewline
        %\cline{2-10}
        \hline
        \textbf{0} & \textbf{4} & \textbf{9} & \textbf{10} & \textbf{11} & \textbf{12} \\
        \hline
    \end{tabular}
    \end{center}

    Mit den Berechtigungen:
    \begin{center}
    \begin{tabular}{ | c | l | }
        \hline
        \textbf{ID} & \textbf{Name} \\
        \hline
        0 & In-App-Purchases \\
        4 & Calender \\
        9 & Pictures/Media/Files \\
        10 & Storage \\
        11 & Camera \\
        12 & Microphone \\
        \hline
    \end{tabular}
    \end{center}

\end{frame}

\begin{frame}{Bewertung der Apps}

    \begin{enumerate}
        \setcounter{enumi}{1}
        \item Berechnung der Gewichtung
    \end{enumerate}

    Besteht aus zwei Teilen:
    \begin{itemize}
        \item Relative häufigkeit von App die diese Berechtigung \textbf{nicht} haben
    \end{itemize}

    \begin{center}
    \begin{tabular}{|*{6}{c|}}
        %\cline{2-10}
        \hline
        0.4 & 0.8 & 0.6 & 0.2 & 0.0 & 0.6 \\
        \hline
    \end{tabular}
    \end{center}

    \begin{itemize}
        \item Bösheit der Berechtigung
    \end{itemize}

    \begin{center}
    \begin{tabular}{|*{6}{c|}}
        %\cline{2-10}
        \hline
        0.1 & 0.6 & 0.1 & 0.1 & 0.9 & 0.9 \\
        \hline
    \end{tabular}
    \end{center}

\end{frame}

\begin{frame}{Bewertung der Apps}

    Diese werden miteinander multipliziert.

    \begin{center}
    \begin{tabular}{|*{6}{c|}}
        \multicolumn{6}{c}{Permissions} \tabularnewline
        %\cline{2-10}
        \hline
        \textbf{0} & \textbf{4} & \textbf{9} & \textbf{10} & \textbf{11} & \textbf{12} \\
        \hline{}
        0.04 & 0.48 & 0.06 & 0.02 & 0.0 & 0.54 \\
        \hline
    \end{tabular}
    \end{center}

    \begin{enumerate}
        \setcounter{enumi}{2}
        \item Füllen der Matrix
    \end{enumerate}

      \begin{center}
      \begin{tabular}{c|c||*{6}{c|}}
        \multicolumn{2}{c}{} & \multicolumn{6}{c}{Permissions} \tabularnewline
        \cline{2-8}
        \multirow{6}*{\rotatebox{90}{Apps}} & ID
    &    \textbf{0} & \textbf{4} & \textbf{9} & \textbf{10} & \textbf{11} & \textbf{12}  \tabularnewline[0 ex]
    %\cline{2-8}
    \hhline{~|=||=|=|=|=|=|=|}
    &    \bfseries 14 & 0.04 & 0.0 & 0.0 & 0.02 & 0.0 & 0.54 \tabularnewline [0 ex]
        \cline{2-8}
    &    \bfseries 42 & 0.0 & 0.48 & 0.06 & 0.0 & 0.0 & 0.0 \tabularnewline [0 ex]
        \cline{2-8}
    &    \bfseries 145 & 0.04 & 0.0 & 0.0 & 0.02 & 0.0 & 0.0 \tabularnewline [0 ex]
        \cline{2-8}
    &    \bfseries 465 & 0.04 & 0.0 & 0.06 & 0.02 & 0.0 & 0.54 \tabularnewline [0 ex]
        \cline{2-8}
    &    \bfseries 1010 & 0.0 & 0.0 & 0.0 & 0.02 & 0.0 & 0.0 \tabularnewline [0 ex]
        \cline{2-8}
      \end{tabular}
      \end{center}

\end{frame}

\begin{frame}{Bewertung der Apps}

    \begin{enumerate}
        \setcounter{enumi}{3}
        \item Aufsummieren der Werte
    \end{enumerate}

      \begin{center}
      \begin{tabular}{c|c||*{6}{c|}|c|}
        \multicolumn{2}{c}{} & \multicolumn{6}{c}{Permissions} & \multicolumn{1}{c}{} \tabularnewline
        \cline{2-9}
        \multirow{6}*{\rotatebox{90}{Apps}} & ID
    &    \textbf{0} & \textbf{4} & \textbf{9} & \textbf{10} & \textbf{11} & \textbf{12} & \(\sum\)  \tabularnewline[0 ex]
    %\cline{2-9}
    \hhline{~|=||=|=|=|=|=|=||=|}
    &    \bfseries 14 & 0.04 & 0.0 & 0.0 & 0.02 & 0.0 & 0.54 & \textbf{0.6} \tabularnewline [0 ex]
        \cline{2-9}
    &    \bfseries 42 & 0.0 & 0.48 & 0.06 & 0.0 & 0.0 & 0.0 & \textbf{0.54} \tabularnewline [0 ex]
        \cline{2-9}
    &    \bfseries 145 & 0.04 & 0.0 & 0.0 & 0.02 & 0.0 & 0.0 & \textbf{0.06} \tabularnewline [0 ex]
        \cline{2-9}
    &    \bfseries 465 & 0.04 & 0.0 & 0.06 & 0.02 & 0.0 & 0.54 & \textbf{0.66} \tabularnewline [0 ex]
        \cline{2-9}
    &    \bfseries 1010 & 0.0 & 0.0 & 0.0 & 0.02 & 0.0 & 0.0 & \textbf{0.02} \tabularnewline [0 ex]
        \cline{2-9}
      \end{tabular}
      \end{center}

\end{frame}

\begin{frame}[fragile]{Bewertung der Apps}

    \begin{enumerate}
        \setcounter{enumi}{4}
        \item Aufteilen in 3 Gruppen mithilfe K-Means
    \end{enumerate}

    \begin{columns}
        \begin{column}{0.5\textwidth}
              \begin{center}
                  \begin{tabular}{c|c|c|}
                    %\multicolumn{2}{c}{} & \multicolumn{6}{c}{Permissions} & \multicolumn{1}{c}{} \tabularnewline
                    \cline{2-3}
                    \multirow{6}*{\rotatebox{90}{Apps}} & ID
                &  \(\sum\) \tabularnewline[0 ex]
                %\cline{2-3}
                \hhline{~|=|=|}
                &    \bfseries \color{red} 14 & \color{red} 0.6 \tabularnewline [0 ex]
                    \cline{2-3}
                &    \bfseries \color{yellow} 42 & \color{yellow} 0.54 \tabularnewline [0 ex]
                    \cline{2-3}
                &    \bfseries \color{green} 145 & \color{green} 0.06 \tabularnewline [0 ex]
                    \cline{2-3}
                &    \bfseries \color{red} 465 & \color{red} 0.66 \tabularnewline [0 ex]
                    \cline{2-3}
                &    \bfseries \color{green} 1010 & \color{green} 0.02 \tabularnewline [0 ex]
                    \cline{2-3}
                  \end{tabular}
              \end{center}
        \end{column}
        \begin{column}{0.5\textwidth} %%
            \begin{itemize}
                \item Gut - Grün
                    \begin{itemize}
                        \item 80 - 120 degree
                    \end{itemize}
                \item Mittel - Gelb
                    \begin{itemize}
                        \item 30 - 79 degree
                    \end{itemize}
                \item Schlecht - Rot
                    \begin{itemize}
                        \item 0 - 29 degree
                    \end{itemize}
            \end{itemize}
        \end{column}
    \end{columns}

    \begin{lstlisting}[language=Python]
# 0 - 100
value = 100 - ((app_values[i] - min_value) * 100.0) / (max_value - min_value)
# min_range - max_range
value = (value * (color_range[1] - color_range[0]) / 100) + color_range[0]
    \end{lstlisting}

\end{frame}

\begin{frame}{Projektmanagement}

    \begin{itemize}
        \textbf{Aufteilung der Arbeit}
         \begin{itemize}
              \item Projektleiter George
              \item Tech-Support Rodion
              \item George, Viktor und Rodion waren für die App zuständig
              \item Simon war für den Webservice zuständig
              \item Robert hat das Data-Mining, GoogleScraper, Scoring und die Datenbank aufgebaut
         \end{itemize}
        \textbf{Zeitmanagement}
            \begin{itemize}
              \item Teamtreffen jede Woche montags um 9:30 Uhr
              \item Treffen mit Herrn Igler mittwochs um 10:00 Uhr
              \item Meilensteine wurden festgelegt
            \end{itemize}
    \end{itemize}
\end{frame}
\begin{frame}{Projektmanagement}
    \begin{itemize}
        \textbf{Kommunikation und Dokumentation}
         \begin{itemize}
              \item Telegramm (Kommunikation)
              \item Slack(Jibble) (Zeiterfassung der Arbeitszeit)
              \item Wiki (Dokumentation des Projektes)
              \item Github (Repository mit all unseren Daten )
        \end{itemize}
    \end{itemize}
\end{frame}
\begin{frame}{Projektmanagement}
    \begin{itemize}
        \textbf{Meilensteine}
         \begin{itemize}
              \item 24.05.2017 Grob-Entwurf unserer App vorstellen
              \item 21.06.2017 App sollte lauffähig sein und mit dem Server funktionieren
              \item App ist final und voll funktionsfähig
        \end{itemize}
    \end{itemize}
\end{frame}
\begin{frame}{Projektmanagement}
\begin{itemize}
        \textbf{Probleme im Projekt}
         \begin{itemize}
              \item Mussten anfangs mit Dummy Daten arbeiten
              \item Daten und Webservice standen am Anfang noch nicht zur Verfügung
              \item Clustering war noch nicht optimiert und hat zu große Cluster generiert
              \item Algorithmus musste angepasst werden
              \item Cluster wurden verkleinert, leider mit Qualitätseinbußen
        \end{itemize}
    \end{itemize}
\end{frame}
\begin{frame}{Projektmanagement}
\begin{itemize}
        \textbf{Vergleich urpsrung Anforderung mit Ergebnis}
         \begin{itemize}
              \item tetststts
              \item testtt
              \item testtt
              \item testttt
              \item xxxxxx
        \end{itemize}
    \end{itemize}
\end{frame}
\begin{frame}{Projektmanagement}

    \textbf{Werdegang der App}

    \begin{columns}[t]
        \column{.5\textwidth}
        \centering
        \includegraphics[width=0.8\textwidth]{img/1.jpg}\\
        \column{.5\textwidth}
        \centering
        \includegraphics[width=0.8\textwidth]{img/2.jpg}
    \end{columns}

\end{frame}
\begin{frame}{Projektmanagement}

    \textbf{Werdegang der App}

    \begin{columns}[t]
        \column{.5\textwidth}
        \centering
        \includegraphics[width=0.8\textwidth]{img/3.jpg}\\
        \column{.5\textwidth}
        \centering
        \includegraphics[width=0.8\textwidth]{img/4.png}
    \end{columns}

\end{frame}

\end{document}


"""
red 0 degree - 29 degree
yellow 30 degree - 79 degree
green 80 degree - 120 degree
"""
